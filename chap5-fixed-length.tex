\chapter{Fixed-Length Items, Parsing (ab)*; Parsing \#s; Parsing \#, ()} \label{ch:fixed-length}
  In this chapter, we explore the Fiat framework with a few example grammars, which we describe how to parse.  Because these rules are so straightforward, they can be handled automatically, in the very first step of the derivation; we will explain how this works, too.

  Recall the grammar for the regular expression \regex{(ab)$^*$}:
  \begin{align*}
    \nt{(ab)$^*$} & \Coloneqq \nt{$\epsilon$}~|~\terminal{a}~\terminal{b}~\nt{(ab)$^*$}
  \end{align*}

  In addition to parsing this grammar, we will also be able to parse the grammar for non-negative parenthesized integers:
  \begin{align*}
    \nt{pexpr} & \Coloneqq \terminal{(}~\nt{pexpr}~\terminal{)}~|~\nt{number} \\
    \nt{number} & \Coloneqq \nt{digit}~\nt{number?} \\
    \nt{number?} & \Coloneqq \epsilon~|~\nt{number} \\
    \nt{digit} & \Coloneqq \terminal{0}~|~\terminal{1}~|~\terminal{2}~|~\terminal{3}~|~\terminal{4}~|~\terminal{5}~|~\terminal{6}~|~\terminal{7}~|~\terminal{8}~|~\terminal{9}
  \end{align*}

  \section{Parsing (ab)*: At Most One Nonterminal}
    The simpler of these grammars is the one for \regex{(ab)$^*$}.  The idea is that if any rule has at most one nonterminal, then there is only one possible split: we assign one character to each terminal, and the remaining characters to the single nonterminal.

    For any given rule, we can compute straightforwardly whether or not this is the case; Haskell-like pseudocode for doing so is:
\begin{verbatim}
has-at-most-one-nt : [Item] -> Bool
has-at-most-one-nt [] = true
has-at-most-one-nt (Terminal _)::xs = has-at-most-one-nt xs
has-at-most-one-nt (Nonterminal _)::xs = has-only-terminals xs

has-only-terminals : [Item] -> Bool
has-only-terminals [] = true
has-only-terminals (Terminal _)::xs = has-only-terminals xs
has-only-terminals (Nonterminal _)::xs = false
\end{verbatim}

  The code for determining the split location is even easier: if the first item of the rule is a terminal, then split at character 1; if the first item of the rule is a nonterminal, and there are $n$ remaining items in the rule, then split $n$ characters before the end of the string.

  \section{Parsing Parenthesized Numbers: Fixed Lengths}
    The grammar for parenthsized numbers has only one rule with multiple nonterminals: the rule for \nt{number}~$\Coloneqq$~\nt{digit}~\nt{number?}.  The strategy here is also simple: because \nt{digit} only accepts strings of length exactly 1, we always want to split after the first character.

    The following pseudocode determines whether or not all strings parsed by a given item are a fixed length, and, if so, what that length is:
    \todo{Colorize code; use macros to ensure it matches the conventions of the rest of the document}
\begin{verbatim}
fixed-length-of : [Nonterminal] -> [Item] -> Maybe Int
fixed-length-of valid_nonterminals [] = just 0
fixed-length-of valid_nonterminals (Terminal _::xs)
  = case fixed-length-of valid_nonterminals xs of
      | just k -> just (1 + k)
      | nothing -> nothing
fixed-length-of valid_nonterminals (Nonterminal nt::xs)
  if nt `in` valid_nonterminals
  then let lengths = map (fixed-length-of (remove nt valid_nonterminals)) (Lookup nt)
       in if all of the elements of lengths are equal to L for some L
          then case L, fixed-length-of valid_nonterminals xs of
                | just L', just k -> just (L' + k)
                | _, _ -> nothing
          else nothing
  else nothing
\end{verbatim}

    We have proven that for any nonterminal for which this method returns \texttt{just }$k$, the only valid split of any string for this rule is at location $k$.  This is the correctness obligation that Fiat demands of us to be able to use this rule.

  \section{Putting It Together}\label{sec:first-mention-first-honing-step}
     Both of these refinement strategies are simple and complete for the rules they handle; if a rule has at most one nonterminal, or if the first element of a rule has a fixed length, then we can't do any better than these rules.  Therefore, we incorporate them into the initial invocation of \verb|start honing parser using indexed representation|.  \editedtext[1]{As described in \autoref{ch:fiat},} to do this, we express the splitter by folding \verb|If| statements over all of the rules of the grammar that are reachable from valid nonterminals.  The \verb|If| statements check equality of the rule against the one we were given, and, if they match, look to see if either of these strategies applies.  If either does, than we return the appropriate singleton value.  If neither applies, then we default to the nondeterministic pick of a list containing all possible valid splits.  The results of applying this procedure without treating any rules specially was shown in \autoref{sec:if-folding}.  \editedtext[1]{The readers interested in precise details can find verbatim code for the results of applying this procedure, including the rules of this chapter, in \appendixautoref{sec:after-if-folding-num-+-paren}.  For the similarly interested readers, the Coq code that computes the goal that the user is presented with, after \tactic{start honing parser using indexed representation}, can be found in \appendixautoref{sec:expanded-fallback-list}.}
     


%
%\begin{verbatim}
%Lemma ComputationalSplitter'
%: FullySharpened (string_spec ab_star_grammar).
%Proof.
%  start honing parser using indexed representation.
%
%  hone method "splits".
%  {
%    simplify parser splitter.
%    finish honing parser method.
%  }
%
%  FullySharpenEachMethodWithoutDelegation.
%  extract delegate-free implementation.
%  simpl; higher_order_reflexivityT.
%Defined.
%\end{verbatim}
%
%\begin{verbatim}
%3 subgoals, subgoal 1 (ID 17)
%
%  ============================
%   Sharpened
%     (ADTRep (string * (nat * nat))
%      { Def Constructor "new" (s: string) : rep :=
%          ret (s, (0, length s)) ,
%        Def Method "to_string" (s : rep, _ : ()) : string :=
%          ret (s, string_of_indexed s) ,
%        Def Method "is_char" (s : rep, ch : Ascii.ascii) : bool :=
%          ret (s, Equality.string_beq (string_of_indexed s) (String ch "")) ,
%        Def Method "get" (s : rep, n : nat) : (option Ascii.ascii) :=
%          ret (s, get n (string_of_indexed s)) ,
%        Def Method "length" (s : rep, _ : ()) : nat :=
%          ret (s, ilength s) ,
%        Def Method "take" (s : rep, n : nat) : () :=
%          ret (fst s, (fst (snd s), min (snd (snd s)) n), ()) ,
%        Def Method "drop" (s : rep, n : nat) : () :=
%          ret (fst s, (n + fst (snd s), NatFacts.minusr (snd (snd s)) n), ()) ,
%        Def Method "splits" (s : rep, p : (item Ascii.ascii *
%                                           production Ascii.ascii)) :
%        (list nat) :=
%          dummy <- {_ : list nat | True};
%          ls <- If [Terminal "a"%char; Terminal "b"%char;
%                   Nonterminal "(ab)*"] =p fst p :: snd p Then
%                ret [1]
%                Else (If [Terminal "b"%char; Nonterminal "(ab)*"] =p
%                         fst p :: snd p Then ret [1]
%                      Else (If [Nonterminal "(ab)*"] =p fst p :: snd p
%                            Then ret [ilength s] Else
%                            ret dummy));
%          ret (s, ls) ,
%        Def Method "rules" (s : rep, p : (productions Ascii.ascii)) :
%        (list nat) :=
%          dummy <- {_ : list nat | True};
%          ls <- If ContextFreeGrammarEquality.productions_beq
%                     Equality.ascii_beq
%                     [[];
%                     [Terminal "a"%char; Terminal "b"%char;
%                     Nonterminal "(ab)*"]] p
%                Then option_rect
%                       (fun _ : option Ascii.ascii => Comp (list nat))
%                       (fun ch : Ascii.ascii =>
%                        If Equality.ascii_beq "a"%char ch Then
%                        ret [1] Else ret dummy) (ret [0])
%                       (get 0 (string_of_indexed s)) Else
%                ret dummy;
%          ret (s, ls)  })%ADT
%
%subgoal 2 (ID 9) is:
% forall idx : StringBound.BoundedString,
% refineADT (Sharpened_DelegateSpecs ?4 idx)
%   (ComputationalADT.LiftcADT
%      (existT (ComputationalADT.pcADT (Sharpened_DelegateSigs ?4 idx))
%         (?6 idx) (?7 idx)))
%subgoal 3 (ID 10) is:
% refineADT (ComputationalADT.LiftcADT (Sharpened_Implementation ?4 ?6 ?7))
%   (ComputationalADT.LiftcADT ?5)
%
%(dependent evars: ?3 using , ?4 open, ?5 open, ?6 open, ?7 open, ?11 using ,)
%\end{verbatim}
%
%
%\begin{verbatim}
%(* hone method "splits". *)
%4 subgoals, subgoal 1 (ID 60)
%
%  r_n : string * (nat * nat)
%  n : item Ascii.ascii * production Ascii.ascii
%  H := ?50 : hiddenT
%  ============================
%   refine
%     (dummy <- {_ : list nat | True};
%      ls <- If [Terminal "a"%char; Terminal "b"%char; Nonterminal "(ab)*"] =p
%               fst n :: snd n Then ret [1]
%            Else (If [Terminal "b"%char; Nonterminal "(ab)*"] =p
%                     fst n :: snd n Then ret [1]
%                  Else (If [Nonterminal "(ab)*"] =p fst n :: snd n
%                        Then ret [ilength r_n] Else
%                        ret dummy));
%      ret (r_n, ls)) (H r_n n)
%
%(* simplify parser splitter. *)
%
%1 focused subgoals (unfocused: 3)
%, subgoal 1 (ID 362)
%
%  r_n : string * (nat * nat)
%  n : item Ascii.ascii * production Ascii.ascii
%  H := ?50 : hiddenT
%  ============================
%   refine
%     (ret
%        (r_n,
%        [If ([Terminal "a"%char; Terminal "b"%char; Nonterminal "(ab)*"] =p
%             fst n :: snd n)
%            || ([Terminal "b"%char; Nonterminal "(ab)*"] =p fst n :: snd n)
%         Then 1
%         Else (If [Nonterminal "(ab)*"] =p fst n :: snd n Then
%               ilength r_n Else 0)])) (H r_n n)
%\end{verbatim}
%
%
%\begin{verbatim}
%  Definition indexed_spec : ADT (string_rep Ascii.ascii) := ADTRep T {
%    Def Constructor "new"(s : String.string) : rep :=
%      ret (s, (0, String.length s)),
%
%    Def Method "to_string"(s : rep, x : unit) : String.string :=
%      ret (s, string_of_indexed s),
%
%    Def Method "is_char"(s : rep, ch : Ascii.ascii) : bool  :=
%      ret (s, string_beq (string_of_indexed s) (String.String ch "")),
%
%    Def Method "get"(s : rep, n : nat) : option Ascii.ascii  :=
%      ret (s, iget n s),
%
%    Def Method "length"(s : rep, x : unit) : nat :=
%      ret (s, ilength s),
%
%    Def Method "take"(s : rep, n : nat) : unit :=
%      ret ((fst s, (fst (snd s), min (snd (snd s)) n)), tt),
%
%    Def Method "drop"(s : rep, n : nat) : unit :=
%      ret ((fst s, (fst (snd s) + n, snd (snd s) - n)), tt),
%
%    Def Method "splits"(s : rep, p : item Ascii.ascii * production Ascii.ascii) : list nat :=
%      fallback_ls <- { ls : list nat
%                     | match fst p with
%                         | Terminal _
%                           => True
%                         | Nonterminal _
%                           => if has_only_terminals (snd p)
%                              then True
%                              else split_list_is_complete G (string_of_indexed s) (fst p) (snd p) ls
%                       end };
%      let ls := (match snd p, fst p with
%                   | nil, _
%                     => [ilength s]
%                   | _::_, Terminal _
%                     => [1]
%                   | _::_, Nonterminal _
%                     => if has_only_terminals (snd p)
%                        then [ilength s - List.length (snd p)]
%                        else fallback_ls
%                 end) in
%      ret (s, ls)
%  }.
%
%
%
%  Definition split_list_is_complete `{HSL : StringLike Char} (str : String) (it : item Char) (its : production Char)
%             (splits : list nat)
%  : Prop
%    := forall n,
%         n <= length str
%         -> production_is_reachable (it::its)
%         -> forall (pit : parse_of_item G (take n str) it)
%                   (pits : parse_of_production G (drop n str) its),
%              Forall_parse_of_item (fun _ nt => List.In nt (Valid_nonterminals G)) pit
%              -> Forall_parse_of_production (fun _ nt => List.In nt (Valid_nonterminals G)) pits
%              -> List.In n splits.
%
%\end{verbatim}
%
%
%
%\begin{verbatim}
%Inductive length_result := same_length (n : nat) | different_lengths | cyclic_length | not_yet_handled_empty_rule.
%
%Coercion collapse_length_result (l : length_result) : option nat
%  := match l with
%       | same_length n => Some n
%       | _ => None
%     end.
%
%Fixpoint length_of_any_production' {Char} (length_of_any_nt : String.string -> length_result)
%         (its : production Char) : length_result
%  := match its with
%       | nil => same_length 0
%       | (Terminal _)::xs => match length_of_any_production' length_of_any_nt xs with
%                               | same_length n => same_length (S n)
%                               | different_lengths => different_lengths
%                               | cyclic_length => cyclic_length
%                               | not_yet_handled_empty_rule => not_yet_handled_empty_rule
%                             end
%       | (Nonterminal nt)::xs => match length_of_any_nt nt, length_of_any_production' length_of_any_nt xs with
%                                   | same_length n1, same_length n2 => same_length (n1 + n2)
%                                   | cyclic_length, _ => cyclic_length
%                                   | _, cyclic_length => cyclic_length
%                                   | different_lengths, _ => different_lengths
%                                   | _, different_lengths => different_lengths
%                                   | not_yet_handled_empty_rule, _ => not_yet_handled_empty_rule
%                                   | _, not_yet_handled_empty_rule => not_yet_handled_empty_rule
%                                 end
%     end.
%
%Lemma length_of_any_production'_ext {Char}
%      f g
%      (ext : forall b, f b = g b)
%      b
%: @length_of_any_production' Char f b = length_of_any_production' g b.
%Proof.
%  induction b as [ | x ]; try reflexivity; simpl.
%  destruct x; rewrite ?IHb, ?ext; reflexivity.
%Qed.
%
%Definition length_of_any_productions'_f
%  := (fun x1 x2
%      => match x1, x2 with
%           | same_length n1, same_length n2 => if Nat.eq_dec n1 n2 then same_length n1 else different_lengths
%           | cyclic_length, _ => cyclic_length
%           | _, cyclic_length => cyclic_length
%           | _, different_lengths => different_lengths
%           | different_lengths, _ => different_lengths
%           | not_yet_handled_empty_rule, _ => not_yet_handled_empty_rule
%           | _, not_yet_handled_empty_rule => not_yet_handled_empty_rule
%         end).
%
%Arguments length_of_any_productions'_f !_ !_ / .
%
%Lemma length_of_any_productions'_f_same_length {n x1 x2}
%: length_of_any_productions'_f x1 x2 = same_length n
%  <-> (x1 = same_length n /\ x2 = same_length n).
%Proof.
%  destruct x1, x2; simpl in *;
%  repeat match goal with
%           | _ => reflexivity
%           | [ H : context[Nat.eq_dec ?x ?y] |- _ ] => destruct (Nat.eq_dec x y)
%           | [ |- context[Nat.eq_dec ?x ?y] ] => destruct (Nat.eq_dec x y)
%           | _ => progress subst
%           | [ H : same_length _ = same_length _ |- _ ] => inversion H; clear H
%           | _ => intro
%           | [ H : _ /\ _ |- _ ] => destruct H
%           | [ |- _ /\ _ ] => split
%           | [ |- _ <-> _ ] => split
%           | _ => congruence
%           | _ => tauto
%         end.
%Qed.
%
%Lemma length_of_any_productions'_f_same_length_fold_right {n x1 x2}
%: fold_right length_of_any_productions'_f x1 x2 = same_length n
%  <-> (x1 = same_length n /\ fold_right and True (map (fun k => k = same_length n) x2)).
%Proof.
%  induction x2; simpl in *; eauto; try tauto.
%  rewrite length_of_any_productions'_f_same_length.
%  rewrite IHx2.
%  tauto.
%Qed.
%
%Definition length_of_any_productions' {Char} (length_of_any_nt : String.string -> length_result)
%           (prods : productions Char)
%: length_result
%  := match prods with
%       | nil => not_yet_handled_empty_rule
%       | p::ps => fold_right
%                    length_of_any_productions'_f
%                    (length_of_any_production' length_of_any_nt p)
%                    (map (length_of_any_production' length_of_any_nt) ps)
%     end.
%
%Lemma length_of_any_productions'_ext {Char}
%      f g
%      (ext : forall b, f b = g b)
%      b
%: @length_of_any_productions' Char f b = length_of_any_productions' g b.
%Proof.
%  unfold length_of_any_productions'.
%  destruct b as [ | ? b]; trivial; [].
%  induction b; try reflexivity; simpl;
%  erewrite length_of_any_production'_ext by eassumption; trivial; [].
%  edestruct (length_of_any_production' g);
%    rewrite IHb; reflexivity.
%Qed.
%
%Definition length_of_any_nt_step {Char} (G : grammar Char) (predata := @rdp_list_predata _ G)
%           (valid0 : nonterminals_listT)
%           (length_of_any_nt : forall valid, nonterminals_listT_R valid valid0
%                                             -> String.string -> length_result)
%           (nt : String.string)
%: length_result.
%Proof.
%  refine (if Sumbool.sumbool_of_bool (is_valid_nonterminal valid0 nt)
%          then length_of_any_productions'
%                 (@length_of_any_nt (remove_nonterminal valid0 nt) (remove_nonterminal_dec _ _ _))
%                 (Lookup G nt)
%          else different_lengths);
%  assumption.
%Defined.
%
%Lemma length_of_any_nt_step_ext {Char G}
%      x0 f g
%      (ext : forall y p b, f y p b = g y p b)
%      b
%: @length_of_any_nt_step Char G x0 f b = length_of_any_nt_step g b.
%Proof.
%  unfold length_of_any_nt_step.
%  edestruct Sumbool.sumbool_of_bool; trivial.
%  apply length_of_any_productions'_ext; eauto.
%Qed.
%
%Definition length_of_any_nt {Char} (G : grammar Char) initial : String.string -> length_result
%  := let predata := @rdp_list_predata _ G in
%     @Fix _ _ ntl_wf _
%          (@length_of_any_nt_step _ G)
%          initial.
%
%Definition length_of_any {Char} (G : grammar Char) : String.string -> length_result
%  := @length_of_any_nt Char G initial_nonterminals_data.
%
%Definition length_of_any_productions {Char} G := @length_of_any_productions' Char (@length_of_any Char G).
%
%
%\end{verbatim}
%
%\begin{verbatim}
%Global Arguments ComputationalSplitter / .
%
%Require Import Fiat.Parsers.ParserFromParserADT.
%Require Import Fiat.Parsers.ExtrOcamlParsers.
%Import Fiat.Parsers.ExtrOcamlParsers.HideProofs.
%
%Time Definition ab_star_parser (str : String.string) : bool
%  := Eval simpl in has_parse (parser ComputationalSplitter) str.
%
%Print ab_star_parser.
%
%Recursive Extraction ab_star_parser.
%
%Definition test0 := ab_star_parser "".
%Definition test1 := ab_star_parser "ab".
%Definition str400 := "abababababababababababababababababababababababababababababababababababababababababababababababababababababababababababababababababababababababababababababababababababababababababababababababababababababababababababababababababababababababababababababababababababababababababababababababababababababababababababababababababababababababababababababababababababababababababababababababababababababababab".
%Definition test2 := ab_star_parser (str400 ++ str400 ++ str400 ++ str400).
%
%Recursive Extraction test0 test1 test2.
%
%\end{verbatim}
