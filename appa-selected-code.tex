\chapter{Selected Coq Code}
  \section{A Fiat Goal After Trivial Rules Are Refined}
      For each grammar, the Fiat framework presents us with goals describing the unimplemented portion of the splitter for this particular grammar.  For example, the goal for the grammar that parses arithmetic expressions involving plusses and parentheses, after taking care of the trivial obligations that we describe in \autoref{ch:fixed-length}, looks like this:\label{sec:after-if-folding-num-+-paren}
\begin{verbatim}
1 focused subgoals (unfocused: 3)
, subgoal 1 (ID 3491)
  
r_n : string * (nat * nat)
n : item ascii * production ascii
H := ?88 : hiddenT
============================
 refine
   (ls <- If ([Nonterminal "pexpr"] =p fst n :: snd n)
             || (([Nonterminal "expr"] =p fst n :: snd n)
             || ([Nonterminal "number"] =p fst n :: snd n)
             || (([Terminal ")"] =p fst n :: snd n)
             || ([Terminal "0"] =p fst n :: snd n)
             || ([Terminal "1"] =p fst n :: snd n)
             || ([Terminal "2"] =p fst n :: snd n)
             || ([Terminal "3"] =p fst n :: snd n)
             || ([Terminal "4"] =p fst n :: snd n)
             || ([Terminal "5"] =p fst n :: snd n)
             || ([Terminal "6"] =p fst n :: snd n)
             || ([Terminal "7"] =p fst n :: snd n)
             || ([Terminal "8"] =p fst n :: snd n)
             || ([Terminal "9"] =p fst n :: snd n)
             || ([Nonterminal "digit"] =p fst n :: snd n)))
          Then ret [ilength r_n]
          Else (If [Nonterminal "pexpr"; Terminal "+";
                   Nonterminal "expr"] =p fst n :: snd n
          Then {splits : list nat
               | ParserInterface.split_list_is_complete
                 plus_expr_grammar
                 (string_of_indexed r_n)
                 (Nonterminal "pexpr")
                 [Terminal "+"; Nonterminal "expr"]
                 splits}
          Else
            ret [If ([Terminal "+"; Nonterminal "expr"]
                        =p fst n :: snd n)
                    || ([Terminal "("; Nonterminal "expr"; Terminal ")"]
                        =p fst n :: snd n)
                    || ([Nonterminal "digit"; Nonterminal "number"]
                        =p fst n :: snd n)
                 Then 1
                 Else (If [Nonterminal "expr"; Terminal ")"]
                             =p fst n :: snd n 
                 Then  pred (ilength r_n)
                 Else (If [Nonterminal "number"]
                             =p fst n :: snd n 
                 Then ilength r_n
                 Else 0))]);
  ret (r_n, ls)) (H r_n n)
\end{verbatim}

   \section{Coq Code for the First Refinement Step} \label{sec:expanded-fallback-list}
       The general code for computing the goal the user is presented with, after \tactic{start honing parser using indexed representation}, is:
\begin{verbatim}
Definition expanded_fallback_list'
           (P : String -> item Ascii.ascii -> production Ascii.ascii
                -> item Ascii.ascii -> production Ascii.ascii
                -> list nat -> Prop)
           (s : T)
           (it : item Ascii.ascii) (its : production Ascii.ascii)
           (dummy : list nat)
: Comp (T * list nat)
  := (ls <- (forall_reachable_productions
               G
               (fun p else_case
                => If production_beq ascii_beq p (it::its) Then
                     (match p return Comp (list nat) with
                        | nil => ret dummy
                        | _::nil => ret [ilength s]
                        | (Terminal _):: _ :: _ => ret [1]
                        | (Nonterminal nt):: p'
                => If has_only_terminals p' Then
                     ret [(ilength s - Datatypes.length p')%natr]
                   Else
                     (option_rect
                       (fun _ => Comp (list nat))
                       (fun (n : nat) => ret [n])
                       { splits : list nat
                       | P
                           (string_of_indexed s)
                           (Nonterminal nt)
                           p'
                           it
                           its
                           splits }%comp
                       (length_of_any G nt))
                      end)
                   Else else_case)
               (ret dummy));
      ret (s, ls))%comp.
\end{verbatim}
